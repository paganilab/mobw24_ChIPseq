% Options for packages loaded elsewhere
\PassOptionsToPackage{unicode}{hyperref}
\PassOptionsToPackage{hyphens}{url}
%
\documentclass[
]{article}
\usepackage{amsmath,amssymb}
\usepackage{iftex}
\ifPDFTeX
  \usepackage[T1]{fontenc}
  \usepackage[utf8]{inputenc}
  \usepackage{textcomp} % provide euro and other symbols
\else % if luatex or xetex
  \usepackage{unicode-math} % this also loads fontspec
  \defaultfontfeatures{Scale=MatchLowercase}
  \defaultfontfeatures[\rmfamily]{Ligatures=TeX,Scale=1}
\fi
\usepackage{lmodern}
\ifPDFTeX\else
  % xetex/luatex font selection
\fi
% Use upquote if available, for straight quotes in verbatim environments
\IfFileExists{upquote.sty}{\usepackage{upquote}}{}
\IfFileExists{microtype.sty}{% use microtype if available
  \usepackage[]{microtype}
  \UseMicrotypeSet[protrusion]{basicmath} % disable protrusion for tt fonts
}{}
\makeatletter
\@ifundefined{KOMAClassName}{% if non-KOMA class
  \IfFileExists{parskip.sty}{%
    \usepackage{parskip}
  }{% else
    \setlength{\parindent}{0pt}
    \setlength{\parskip}{6pt plus 2pt minus 1pt}}
}{% if KOMA class
  \KOMAoptions{parskip=half}}
\makeatother
\usepackage{xcolor}
\usepackage[margin=1in]{geometry}
\usepackage{graphicx}
\makeatletter
\def\maxwidth{\ifdim\Gin@nat@width>\linewidth\linewidth\else\Gin@nat@width\fi}
\def\maxheight{\ifdim\Gin@nat@height>\textheight\textheight\else\Gin@nat@height\fi}
\makeatother
% Scale images if necessary, so that they will not overflow the page
% margins by default, and it is still possible to overwrite the defaults
% using explicit options in \includegraphics[width, height, ...]{}
\setkeys{Gin}{width=\maxwidth,height=\maxheight,keepaspectratio}
% Set default figure placement to htbp
\makeatletter
\def\fps@figure{htbp}
\makeatother
\setlength{\emergencystretch}{3em} % prevent overfull lines
\providecommand{\tightlist}{%
  \setlength{\itemsep}{0pt}\setlength{\parskip}{0pt}}
\setcounter{secnumdepth}{-\maxdimen} % remove section numbering
\ifLuaTeX
  \usepackage{selnolig}  % disable illegal ligatures
\fi
\IfFileExists{bookmark.sty}{\usepackage{bookmark}}{\usepackage{hyperref}}
\IfFileExists{xurl.sty}{\usepackage{xurl}}{} % add URL line breaks if available
\urlstyle{same}
\hypersetup{
  pdftitle={Workshop Introduction},
  pdfauthor={Carolina Dossena},
  hidelinks,
  pdfcreator={LaTeX via pandoc}}

\title{Workshop Introduction}
\author{Carolina Dossena}
\date{last-modified}

\begin{document}
\maketitle

Welcome! This workshop has been created as part of the
\href{https://www.unimi.it/it/corsi/insegnamenti-dei-corsi-di-laurea/2024/molecular-biology-applied-biotechnology}{Unimi
course} ``\emph{Molecular Biology applied to Biotechnology}''. The aim
is to introduce you to the world of bioinformatics and to some of the
most important concepts related to the \textbf{analysis of Next
Generation Sequencing (NGS) data}.

In particular, you will be guided through the main steps of a
\textbf{ChIP-seq experiment}, starting from the experimental design and
its major challenges and then diving into a ChIP-seq analysis workflow!
The hands-on part is based on \texttt{R} and for this reason an
essential introduction to this programming language will be provided as
well.

The dataset used in this workshop is taken from
\href{https://www.nature.com/articles/s41467-021-22544-y\#MOESM1}{our
study} published on \emph{Nature Communications} on 2021,
``\emph{Epigenomic landscape of human colorectal cancer unveils an
aberrant core of pan-cancer enhancers orchestrated by YAP/TAZ}''. Part
of the adventure will be dedicated to trying to reproduce some relevant
analyses and plots published as results!

\hypertarget{learning-objectives}{%
\section{Learning Objectives}\label{learning-objectives}}

\begin{enumerate}
\def\labelenumi{\arabic{enumi}.}
\tightlist
\item
  Get an overview of the ChIP-seq \textbf{technology}
\item
  Understand the best practices for the \textbf{design} of a ChIP-seq
  experiment
\item
  Get familiar with the steps of a standard \textbf{ChIP-seq data
  analysis pipeline}
\item
  Know the most useful \textbf{public resources} for the analysis and
  exploration of \textbf{genomic data}
\end{enumerate}

\hypertarget{workshop-schedule}{%
\section{Workshop Schedule}\label{workshop-schedule}}

This workshop is intended as a three-day tutorial. Each day will be
dedicated to specific activities:

\hypertarget{day-1}{%
\subsubsection{Day 1}\label{day-1}}

\begin{itemize}
\tightlist
\item
  Setup Rstudio or Posit Cloud
\item
  Get familiar with R and create our R environment
\item
  Learn about ChIP-seq experimental design and technology
\item
  Describe the main steps of ChIP-seq (and NGS) data \emph{core}
  processing
\item
  Download the dataset
\end{itemize}

\hypertarget{day-2}{%
\subsubsection{Day 2}\label{day-2}}

\begin{itemize}
\tightlist
\item
  Data normalization with \texttt{edgeR}
\item
  Diagnostic and exploratory analysis on the data
\item
  Differential analysis for ChiP-seq data
\item
  Visualization of the results
\end{itemize}

\hypertarget{day-3}{%
\subsubsection{Day 3}\label{day-3}}

\begin{itemize}
\tightlist
\item
  Downstream analyses on an interesting subset of data, including:

  \begin{itemize}
  \tightlist
  \item
    Gene ontology with \texttt{gProfiler}
  \item
    Motif Analysis with \texttt{MEME}
  \end{itemize}
\item
  Explore different online resources dedicated to ChIP-seq data and NGS
  data analysis
\end{itemize}

\hypertarget{the-teaching-board}{%
\section{The Teaching Board}\label{the-teaching-board}}

🧑🔬 \texttt{Jacopo\ Arrigoni}
(\href{mailto:jacopo.arrigoni@ifom.eu}{\nolinkurl{jacopo.arrigoni@ifom.eu}})

👩💻 \texttt{Carolina\ Dossena}
(\href{mailto:carolina.dossena@ifom.eu}{\nolinkurl{carolina.dossena@ifom.eu}})

\hypertarget{credits}{%
\section{Credits}\label{credits}}

This workshop was inspired by other tutorials on ChIP-seq data analysis
(\href{https://www.bioconductor.org/help/course-materials/2016/CSAMA/lab-5-chipseq/Epigenetics.html}{the
Bioconductor course}, the teaching material from the
\href{https://github.com/hbctraining/Intro-to-ChIPseq/blob/master/schedule/3-day.md}{HBC
training} and the
\href{http://biocluster.ucr.edu/~rkaundal/workshops/R_feb2016/ChIPseq/ChIPseq.html}{tutorial
from UCR}. \textbf{Mattia Toninelli}
(\href{mailto:mattia.toninelli@ifom.eu}{\nolinkurl{mattia.toninelli@ifom.eu}})
helped with the development of this site. The design of the analyses and
the codes have been generated together with \textbf{Michaela Fakiola}
(\href{mailto:michaela.fakiola@ifom.eu}{\nolinkurl{michaela.fakiola@ifom.eu}}).

\hypertarget{license}{%
\section{License}\label{license}}

All of the material in this course is under a
\href{https://creativecommons.org/licenses/by/4.0/}{Creative Commons
Attribution license} (\emph{CC BY 4.0}) which permits unrestricted use,
distribution, and reproduction in any medium, provided the original
author and source are credited.

\end{document}
